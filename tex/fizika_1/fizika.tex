\documentclass[10pt]{article}
\usepackage{amssymb}
\usepackage{amsmath}
\usepackage{subfiles}
\usepackage[usenames,dvipsnames]{xcolor}
\usepackage{indentfirst}
\usepackage{microtype}
\usepackage{graphicx}
\usepackage{geometry}
	\geometry{a4paper, total={170mm,257mm}, left=20mm, top=20mm, }

\renewcommand*\contentsname{Tartalomjegyzék}

\AddToHook{cmd/section/before}{\clearpage}

\begin{document}
\begin{titlepage}
	\centering \vfill
	{\textsc{Budapesti Műszaki és Gazdaságtudományi Egyetem} \par} \vspace{7cm}
	{\huge\bfseries Fizika 1\par} \vspace{0.5cm}
	{\large \textsc{Kidolgozott szóbeli tételsor}\par} \vspace{1.5cm}
	{\Large\itshape Készítette: Illyés Dávid Gyula\par} \vfill

	\noindent\fbox{%
    	\parbox{140mm}{
			\color{red}\textbf{Ez  a jegyzet nagyon hasonlóan van struktúrálva az előadás jegyzetekhez és fő célja, hogy olyan módon adja át a "A Programozás Alapjai 1" nevű tárgy anyagát, hogy az teljesen kezdők számára is könnyen megérthető és megtanulható legyen. }
   		}
	}

	\vfill {\large \today\par}
\end{titlepage}
\tableofcontents
\addtocontents{toc}{~\hfill\textbf{Oldal}\par}

	\section{Kinematika}

		\subfile{sections/1/kinematika.tex}

	\section{Dinamika alapjai}

		%\subfile{sections/2/minktgffa.tex}

	\section{Gyorsuló vonatkoztatási rendszerek}

		%\subfile{sections/3/bfsdijkstra.tex}

	\section{Mozgások energetikája}

		%\subfile{sections/4/fordfloyddfs.tex}

	\section{Pontrendszerek dinamikája}

		%\subfile{sections/5/eulerhamilton.tex}

	\section{Forgó mozgás dinamikája}

		%\subfile{sections/6/sikgrafok.tex}

	\section{Merev testek forgómozgása}

		%\subfile{sections/7/gausseliminacio.tex}

	\section{Rezgések}

	\section{Hullámok}

	\section{Hőten alapjai}

	\section{Körfolyamatok, hőerőgépek}

	\section{Cseppfolyós és szilárd anyagok hőtana}


\end{document}
