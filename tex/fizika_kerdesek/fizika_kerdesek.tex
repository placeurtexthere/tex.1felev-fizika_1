\documentclass[10pt]{article}
\usepackage{amssymb}
\usepackage{amsmath}
\usepackage{subfiles}
\usepackage[usenames,dvipsnames]{xcolor}
\usepackage{indentfirst}
\usepackage{microtype}
\usepackage{graphicx}
\usepackage{stackrel}
\usepackage{geometry}
	\geometry{a4paper, total={170mm,257mm}, left=20mm, top=20mm, }

\renewcommand*\contentsname{Tartalomjegyzék}

\AddToHook{cmd/section/before}{\clearpage}

\begin{document}
\begin{titlepage}
	\centering \vfill
	{\textsc{Budapesti Műszaki és Gazdaságtudományi Egyetem} \par} \vspace{7cm}
	{\huge\bfseries Fizika 1\par} \vspace{0.5cm}
	{\large \textsc{Mondat kiegészítő feladatok gyűjteménye 2016 - 2023}\par} \vspace{1.5cm}
	{\Large\itshape Készítette: Illyés Dávid Gyula\par} \vfill

	\noindent\fbox{%
    	\parbox{140mm}{
			Ez a jegyzet a Fizika 1 tárgy ZH-kban és Vizsgákban lévő mondat kiegészítős feladatokat gyűjti össze, a könnyebb felkészülés könnyítése érdekében. Minden sorban a \underline{\textbf{félkövér aláhúzott}} szó/kifejezés/képlet a kiegészített része a mondatnak. Néhány esetben, amikor képletek a megoldások valamiért nem vállik félkövérré a szöveg, de ugyan úgy alá van húzva! Elnézést kérek a kellemetlenségekért.
   		}
	}

	\vfill {\large \today\par}
\end{titlepage}
\tableofcontents
\addtocontents{toc}{~\hfill\textbf{Oldal}\par}

	\section{2016}	

		\subfile{sections/2016/2016.tex}

	\section{2017}

		\subfile{sections/2017/2017.tex}

	\section{2018}

		\subfile{sections/2018/2018.tex}

	\section{2019}

		\subfile{sections/2019/2019.tex}

	\section{2021}

		\subfile{sections/2021/2021.tex}

	\section{2022}

		\subfile{sections/2022/2022.tex}

	\section{2023}

		\subfile{sections/2023/2023.tex}

\end{document}
