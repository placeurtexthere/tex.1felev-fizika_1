\documentclass[../../fizika_kerdesek.tex]{subfiles}

\begin{document}

    \subsection{2016.11.10 - Nagy ZH}

        \begin{enumerate}
            \item A tehetetlenségi törvénye csak \underline{\textbf{inerciarendszer}}-ben érvényes.
            \item Függőlegesen elhajítunk egy labdát, mely $h$ magasságban emelkedik, majd visszaesik és elkapjuk. Az elmozdulás nagysága \underline{\textbf{nulla}}.
            \item A ferde hajítás során a test \underline{\textbf{gyorsulás}} vektora  mindvégig állandó.
            \item Lejtőre helyezett testre ható tartóerő a lejtő hajlásszögének \underline{\textbf{cosinusával}} arányos.
            \item Az $F_{ts}$ tapadási súrlódási erő és a felületeket összenyomó $F_t$ erő között az alábbi összefüggés áll fenn: \underline{\textbf{$F_{ts} \le F_t \cdot \mu_0$}} ahol $\mu_0$ a \underline{\textbf{tapadási súrlódási együttható}}.
            \item Egy elütött jégkorong lassulásának nagysága 0,5 $m/s^2$. A jég és a korong közti csúszási súrlódási együttható értéke közelítőleg: \underline{\textbf{0,05}}.
            \item A Föld déli féltekén északi irányban közlekedő vonatokra \underline{\textbf{nyugatra}} mutató Coriolis-erő hat.
            \item Lefelé gyorsuló liftben a lifthez képest nyugvó testre ható gravitációs erő \underline{\textbf{ugyanakkora}}, mint egy nyugvó liftben elhelyezett testre ható gravitációs erő.
            \item Egy Hooke-törvénynek engedelmeskedő rugalmas erőtérben mozgó test potenciális energiáját az alábbi összefüggés adja meg: \underline{\textbf{$\dfrac{1}{2}kx^2$}} ahol $k$ a \underline{\textbf{rugóállandó}}.
            \item A Nap gravitációs erőtérnek Földön végzett munkája egy év alatt \underline{\textbf{nulla}}.
            \item A \underline{\textbf{munkatétel}} értelmében a testre ható erők eredőjének munkája egyenlő a test mozgási energiájának megváltozásával.
            \item Konzervatív erőtérben mozgó test \underline{\textbf{mechanikai energiája}} megmarad.
        \end{enumerate}

    \underline{\textbf{}}

\end{document}