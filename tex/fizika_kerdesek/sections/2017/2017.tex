\documentclass[../../fizika_kerdesek.tex]{subfiles}

\begin{document}

    \subsection{2017.11.09. - Nagy ZH}

        \begin{enumerate}
            \item A mechanikát leíró fizikai mennyiségek \underline{\textbf{3}} darab SI alapmennyiségből származtathatóak.
            \item A mozgás kezdő- és végpontja közöti pályagörbe hosszát \underline{\textbf{útnak}} nevezzük.
            \item A ferdén elhajított test vízszintes tengelyre vetített mozgása \underline{\textbf{egyenes vonalú egyenletes mozgás}}.
            \item Súrlódásmentes lejtőre helyezett test gyorsulása a lejtő hajlászögének \underline{\textbf{szinuszá}} -val arányos.
            \item Vízszintes úton gépkocsi gyorsít. Az aztót a \underline{\textbf{tapadási súrlódási}} erő gyorsítja. 
            \sloppy\item A nyugvó tengerek vízfelülete merőleges a gravitációs erő, valamint a Föld forgásából származó \underline{\textbf{centrifugális erő}} szuperpozíciójára.
            \item A Föld északi féltekén északról délre fújó szelekre \underline{\textbf{nyugat felé}} mutató Coriolis-erő hat.
            \item A levegő lefelé eső ejtőernyőn végzett munkája \underline{\textbf{negatív}} előjelű.
            \item A nehézségi erőtérbe helyezett test potenciális energiájának megadására használt $E_{pot}=mgh$ összefüggés csak azon feltevés mellett érvényes, ha a negézségi erőteret \underline{\textbf{homogénnek}} tekintjük.
            \item Egy tisztán gördülő roller első kereke kétszer akkora, mint a hátsó. Az elsp kerék kerületi pontjainak centripetális gyorsulása \underline{\textbf{1/2}} -szerese, mint a hátsó keréké.
            \item A munkatétel értelmében a testre ható erők eredőjének munkája egyenlő a \underline{\textbf{test kinetikus energiájának megváltozásával}}.
            \item Egy elektromos autó 100 km/h-ról 50km/h-ra lassít, majd megáll. A fékezés két szakasza alatt felszabadult energiát visszatáplálja akkumlátoraiba. A fékezés első, illetve második szakasza alatt visszatáplált enerdiák hányada \underline{\textbf{3:1}}.
        \end{enumerate}

    \subsection{2017.11.23. - Pót Nagy ZH}

        \begin{enumerate}
            \item Vektorok skaláris szorzata arányos a két vektor által közbezárt szög \underline{\textbf{cosinusával}}.
            \item Eötvös Lóránd mérései szerint a testek \underline{\textbf{súlyos}} és \underline{\textbf{tehetetlen tömege}} 7 tizedesjegy pontossággal megegyezik.
            \item Egyenletes körmozgás szögsebességének és fordulatszámának hányadosa \textbf{\underline{$2\pi$}}.
            \item Ferde hajítás során a test \underline{\textbf{gyorsulás}} -vektora állandó.
            \item A Föld felszíne felett R magaságban a gravitációs gyorsulás értéke \underline{\textbf{1/4}} -szerese a Föld felszínén mért gravitációs gyorsulásnak. (R a Föld sugara).
            \item Vízszintes talajon nyugszik egy $m$ tömegű test. A testet vízszintes $F$ erővel húzzuk, de a test nem modul meg. A talaj és a test között mérhető tapadási súrlódási együttható $\mu_0$. A tapadási súrlódási erő nagysága: \underline{\textbf{F}}.
            \item A foucault-inga lengési síkját a \underline{\textbf{Coriolis}} -erő változtatja meg. 
            \item A munka, valamint a munkavégzéshez szükséges idő hányadosát \underline{\textbf{teljesítménynek}} nevezzük.
            \item A rugóban tárolt energia arányos a rugó megnyúlásának \underline{\textbf{2}} hatványával.
            \item Ha egy erőtér nem konzervatív, nem érvényes a \underline{\textbf{mechanikai energia megmaradás}} törvénye. 
            \item Pontrendszer \underline{\textbf{tömegközéppontjának}} gyorsulása arányos a pontrendszerre ható külső erők eredőjével.
            \item Pontrendszer impulzusának \underline{\textbf{időegységenkénti megváltozása}} arányos a pontrendszerre ható külső erők eredőjével.
        \end{enumerate}

    \subsection{2017.12.11. - Pót Pót Nagy ZH}

        \begin{enumerate}
            \item \underline{\textbf{Inerciarendszerben}} a magára hagyott testek megőrzik mozgásállapotukat. 
            \item Ha azt szeretnénk, hogy egy test háromszor olyan hosszú ideig essen szabadon, \underline{\textbf{kilencszer}} nagyobb magasságból kell leejtenünk.
            \item A ferde hajítás pályájának tetőpontján a test sebességének \underline{\textbf{függőleges öszetevője}} zérus.
            \item Könnyen gördülő bicikli állandósult sebességgel gurul le egy lejtőn. A biciklire ható közegellenállási erő egyensúlyt tart a nehézségi erő \underline{\textbf{lejtővel párhuzamos összetevőjével}}.
            \item A Föld felszínén a legnagyobb centrifugális erő a/z \underline{\textbf{egyenlítőn}} elhelyezett testekre hat.
            \item A föld déli féltekén déli irányban közlekedő vonatokra \underline{\textbf{kelet felé}} mutató Coriolis-erő hat.
            \item Elhajított kiterjedt test \underline{\textbf{tömegközéppontja}} parabola pályán mozog.
            \item Konzervatív erőtér munkája nem függ az erőtérben mozgó test által megtett útról, csak a mozgás \underline{\textbf{kezdő- és végpontjának}} helyzetétől.
            \item Egy erő munkája az erő és az elmozdulás által bezárt szög \underline{\textbf{cosinusával}}.
            \item Adott sebességű autó megállításakor a fékbetétek által végzett munka \underline{\textbf{ugyanakkora}}, ha a fékutat felére csökkentjük.
            \item Súlyerőnek hívjuk azt az erőt, amellyel a test \underline{\textbf{az alátámasztást nyomja}}. A súlyerő \underline{\textbf{az alátámasztás}} -ra/re hat.
            \item \underline{\textbf{Konzervatív}} erőtérben mozgó test mechanikai energiája megmarad.
        \end{enumerate}

    \subsection{2017.12.20. - Vizsga}

        \begin{enumerate}
            \item A testek mozgásállapot változó hatás ellenében tanúsított ellenhatást a \underline{\textbf{(tehetetlen) tömeg}} nevű fizikai mennyiséggel jellemezzük.
            \item Rugalmatlan ütközés előtt a testek mechanikai energiáinak összege mindig \underline{\textbf{nagyobb}} mint ütközés után.
            \item Inerciarendszerekben igaz a \underline{\textbf{tehetetlenség}} törvénye.
            \item Egy hullámvasút egy függőleges síkú hurok legfelső pontján mozog, az utasok mégsem esnek ki. Ekkor a jármű \underline{\textbf{centripetális}} gyorsulása nagyobb, mint \underline{\textbf{g}}.
            \item Tömegpontrendszerben teljes impulzusa megmarad, ha a tömegpontrendszerre ható külső \underline{\textbf{erők eredője nulla}}.
            \item \underline{\textbf{Centrális}} erőtérben mozgó tömegpontra ható erő mindig párhuzamos egy adott vonatkoztatási pontból a tömegponthoz húzott sugárral.
            \item Kepler III. törvénye értelmében a bolygópályák nagytengelyeinek \underline{\textbf{köbei}} úgy aránylanak egymáshoz, mint a keringési idők \underline{\textbf{négyzetei}}.
            \item Hőtágulás következtében egy forgó test minden mérete arányosan megnő $\gamma$-szorosára. A tehetetlenségi nyomatéka ekkor \underline{\textbf{$\gamma^2$}} szorosára nő.
            \sloppy\item A munkatétel értelmében a testre ható erők munkája egyenlő a test \underline{\textbf{kinetikus energiájának megváltozásával}}.
            \item A mindkét végén nyitott síp alapharmonikusának, mint állóhullámnak a csomópontja a síp \underline{\textbf{közepén}} található.
            \item Mechanikus hullámokat terjesztő közeg minden egyes pontja \underline{\textbf{rezgő}} mozgást végez.
            \item \underline{\textbf{Izochor}} folyamatokban a gáz nyomása egyenesen arányos a hőmérséklettel.
            \item Izochor folyamat esetén a \underline{\textbf{gáz belső energiájának megváltozása}} megegyezik a gázzal közölt hőmennyiséggel.
            \item A \underline{\textbf{termodinamika II. főtételének}} értelmében nem konstruálható olyan hőerőgép, mely a befektetett hőt teljes egészében mechanikai munkává tudná alakítani.
            \item Az \underline{\textbf{intenzív}} állapotjellemzők kölcsönhatás során kiegyenlítődnek.
        \end{enumerate}


    \underline{\textbf{}}

\end{document}