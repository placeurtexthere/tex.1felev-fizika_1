\documentclass[../../fizika_kerdesek.tex]{subfiles}

\begin{document}

    \subsection{2018.01.03. - Vizsga}

        \sloppy{
            \begin{enumerate}
                \item A \underline{\textbf{sebesség}} egységnyi idő alatt bekövetkezett megváltozását gyorsulásnak nevezzük.
                \item Egy $m$ és egy $2m$ tömegű  bolygó gravitációs kölcsönhatásába lépnek egymással. A $2m$ tömegű bolygóra \underline{\textbf{ugyanakkora}} erő hat, mint az $m$ tömegű bolygóra.
                \item Egy $h$ magasságú, súrlódásmentes lejtőn lecsúsztatott test \underline{\textbf{ugyanakkora}} sebességgel érkezik a lejtő aljára, mint amekkora egy $h$ magasságból szabadon ejtett test végsebessége.
                \item Konzervatív erőtérben mozgó tömegpont \underline{\textbf{mechanikai energiája}} állandó. 
                \item A röptében szétrobbanó tüzijáték darabkái alátal alkotott tömegpontrendszer \underline{\textbf{tömegközppontja}} egy ferde hajítás pályáján mozog.
                \item Egy homogén tömegeloszlású rúd rúdra merőleges tengelyre vonatkoztatott tehetetlenségi nyomatéka akkor a legkisebb, ha a tengely a rúd \underline{\textbf{tömegközéppontján}} halad át.
                \item Ugyanazon lejtő tetejérpl kezdősebesség nélkül gurítunk le egy tömör hengert, valamint egy ugyanakkora tömegű és sugarú csődarabot. A \underline{\textbf{tömör henger}} ér le hamarabb a lejtő aljára.
                \item Kepler II. törvénye értelmében a napból a bolygóhoz húzott sugár \underline{\textbf{egyenlő időközönként egyenlő !VALAMI! !VALAMI!}}
                \item Az állóhullám két \underline{\textbf{ellentétes}} irányban terejdő haladó hullám interferenciájaként alakul ki.
                \item Mindkét végén rögzített húr alaphangja \underline{\textbf{ugyanakkora}} frekvenciájú, mint egy ugyanolyan hosszú, mindkét végén nyitott síp alaphangja.
                \item Rezonancia esetén a gerjesztett rendszer rezgése, valamint a gerjesztő rezgés közötti fáziskülönbség \underline{\textbf{$\pi/2$}}.
                \item Hőerőgépekben lezajló körfolyamatok $P-V$ diagramon ábrázolva olyan zárt görbéket alkotnak, melyek körüljárási iránya az óramuató járásával \underline{\textbf{megegyező}} irányú.
                \item A Carnot-gép hatásfoka elvileg 100\% -hoz tart, ha a hideg hőtartály hőmérséklete \underline{\textbf{0}} Kelvin fokhoz tart.
                \item Izo \underline{\textbf{term}} állapotváltozás során a gáz belső energiája nem változik.
                \item Izo \underline{\textbf{chor}} állapotváltozás során a gáz belső energiájának megváltozása megegyezik a gázzal közölt hővel.
            \end{enumerate}}

    \subsection{2018.01.10. - Vizsga}

        \sloppy{
            \begin{enumerate}
                \item A tehetetlenség törvénye \underline{\textbf{inerciarendszerek}}-ben érvényes.
                \item Egy $2h$ magasságból ejtett test \underline{\textbf{$\sqrt{2}$}}-szer annyi ideig esik szabadon, mint egy $h$ magasságból ejtett test.
                \item Newton törvényei értelmezhetők gyorsuló vonatkoztatási renszerekben is, ha bevezetjük a \underline{\textbf{tehetetlenségi erőket}}.
                \item Egy erőtér \underline{\textbf{homogén}}, ha a tér minden pontjában ugyanakkora erő hat.
                \item Pontrendszer tömegközéppontjának gyorsulását a pontrendszerben ébredő belső erők \underline{\textbf{nem befolyásolják}}.
                \item Billenő platójú teherautó rakománya akkor csúszik meg, amikor a rakományra ható nehézségi erő \underline{\textbf{plató síkjával párhuzamos komponense}} nagyobb, mint a tapadási súrlódási erő.
                \item Adott bolygó felszínén értelemezett I. kozmikus sebességre gyorsított test képes arra, hogy \underline{\textbf{a bolygó felszíne közelében körpályára álljon}}.
                \item Matematikai inka hosszát megduplázzuk. A lengési \underline{\textbf{$\sqrt{2}$-esére}} változik.
                \item Az egydimenziós hullámegyenlet megoldása egy \underline{\textbf{két}} -változós függvény.
                \item A rezonancia-frekvenciánál jóval alacsonyabb frekvenciával gerjesztett rendszer rezgésének fázisa, valamint a gerjesztő rezgés fázisa között \underline{\textbf{0 fok}} különbség van.
                \item Hőszivattyúkban lezajló körfolyamatok $P-V$ diagramon ábrázolva olyan zárt görbéket alkotnak, melyek körüljárási iránya az óramutató járásával \underline{\textbf{ellentétes}} irányú.
                \item A Carnot-gép hatásfoka elvileg 100\%-hoz tart, ha a meleg hőtartály hőmérséklete \underline{\textbf{végtelenhez}} tart.
                \item Egy ideális gáz adiabatikus tágulása \underline{\textbf{alacsonyabb}} véghőmérsékletet eredményez, minha ugyanazon gázt izoterm folyamat során tágítjuk ugyanakkora térfogatúra.
                \item A hőtan \underline{\textbf{második}} főtételéből következik, hogy két hőtartállyal rendelkező ciklikus hőerőgépek közül a Carnot-gép hatáshoka a legnagyobb.
            \end{enumerate}}

    \subsection{2018.11.09. - Nagy ZH}

        \sloppy{
            \begin{enumerate}
                \item A fizikai mennyiség a mérőszámból és a \underline{\textbf{mértékegységből}} áll. 
                \item Azokat a mennyiségeket, melyeknek nagysága és \underline{\textbf{iránya}} is van, vektormennyiségeknek nevezzük.
                \item Egy testet függőlegesen elhajítunk a talajról $v$ kezdősebességgel, egy másikat $45^{\circ}$-os szög alatt $2v$ sebességgel. A \underline{\textbf{függőlegesen}} elhajított test ér előbb földet.
                \item Lejtőre helyezünk egy hasábot, de az nem csúszik le. A hasábra ható tapadási súrlódási erő nagysága \underline{\textbf{ugyanakkora}}, mint a nehézségi erő lejtővel párhuzamos komponense.
                \item A Hooke-törvény értelmében a \underline{\textbf{rugó}} a kitérítéssel arányos, azzal ellentétes irányú erőt fejt ki.
                \item Gyorsuló vonatkoztatási rendszerekben \underline{\textbf{tehetetlenségi}} erőket definiálunk annak érdekében, hogy a Newton törvényeket az inerciarendszerekben megszokott alakban tudjuk felírni.
                \item A centrifugális erő a forgó vonatkoztatási rendszer szögsebességének \underline{\textbf{második}} hatványával arányos.
                \item Egy tömegpontra $F$ erő hat, miközben a test elmozdul. Az erő munkája nulla, ha az erő és az elmozdulás-vektor \underline{\textbf{merőleges egymásra}}.
                \item Ha egy erőtérben mozgó testre érvényes a mechanikai energia megmaradás törvénye, akkor az erőtér \underline{\textbf{konzervatív}}.
                \item A munkatétel értelmében a testre ható erők eredőjének munkája egyenlő a \underline{\textbf{test kinetikus energiájának megváltozásával}}.
                \item \underline{\textbf{Tisztán gördülő}} kerék talajjal érintkező pontjának pillanatnyi sebessége nulla.
                \item Egy erőteret \underline{\textbf{homogénnak}} nevezünk, ha az erő vektora a tér minden pontjában ugyanakkora.
            \end{enumerate}}

    \subsection{2018.11.20. - Pót Nagy ZH}

        \sloppy{
            \begin{enumerate}
                \item A mechanika törvényeiben előforduló három SI alapmennyiség mértékegységeit a következőképp jelöljük: \underline{\textbf{m  kg  s}}.
                \item A tehetetlenség törvénynek értelmében egy tömegpont mindaddig megőrzi mozgásállapotát, \underline{\textbf{amíg nem lép kölcsönhatásba más testel}}.
                \item Egy ferdén felfelé elhajított test sebességvektora és gyorsulásvektora a pálya \underline{\textbf{kezdő}} pontján zár be egymással a legnagyobb szöget.
                \item Egy $\alpha$ hajlásszögű lejtőn ellenállás nélkül gördül le egy tartálykocsi. A tartályban lévő folyadék felszíne a lejtő síkjával \underline{\textbf{0 fokos}} szöget zár be.
                \item Egy rugó által kifejtett erőt abrázoljuk a rugó megnyúlásának függvényében. A rugóban tárolt energiát a függvény \underline{\textbf{görbe alatti területe}} adja meg.
                \item Egy repülőgép vízszintes pályán közelít a déli serk felé. A Coriolis-erő a pilóta \underline{\textbf{bal}} kezének irányába mutat.
                \item Egy adott forgó vonatkoztatási rendszerben lévő tömegpontra ható centrifugális erő csak a tömegpont helyzetétől függ, ezért a centrifugális erőt \underline{\textbf{erőtérnek}} nevezzük.
                \item Egy testet $F$ erő gyorsít fel álló helyzetből $v$ sebességre. A test mozgási energiája megegyezik az erő \underline{\textbf{munkájával}}.
                \item Egy $3m$ és egy $m$ tömegű gyrmagolyót helyezünk el egymástól adott távolságra. A nagyobbik golyóból lecsípünk $m$ tömeget, és hozzágyúrjuk a kissebbik golyóhoz. A két golyó közti gravitációs kölcsönhatás mértéke \underline{\textbf{nő}}.
                \item Vízszintes talajon tisztán gördülő kerék talajtól legtávolabbi pontjának sebessége \underline{\textbf{kétszer}} akkora, mint a tengely sebessége.
                \item Egy $k$ rugóállandójú rugó mindkét végét $F$ erővel húzzuk, egymással ellentétes irányban. A rugó megnyúlását az \underline{\textbf{$X = \frac{F}{k}$}} összefüggés adja meg.
                \item Egy virágcserép kiesik egy 4. emeleti ablakból. A cserép mozgási energiája a földszinten 4-szer akkora, mint a \underline{\textbf{3.}} emeleten.
            \end{enumerate}}

    \subsection{2018.12.13. - Pót Pót Nagy ZH}

        \sloppy{
            \begin{enumerate}
                \item Az inercia-rendszerek egymáshoz képest \underline{\textbf{nyugalomban}} vannak, vagy \underline{\textbf{egyenes vonalú egyenletes}} mozgást végeznek. 
                \item A ferde hajítás felgontható egy függőleges irányú \underline{\textbf{egyenletesen változó}} valamint egy vízszintes irányú \underline{\textbf{egyenletes}} mozgásra. 
                \item A ferde hajítás pályájának tetőpontján a test pillanatnyi sebességének függőleges komponense \underline{\textbf{nulla}}.
                \item A Föld felszínén az \underline{\textbf{északi vagy déli sarokon}} elhelyezett, nyugalomban lévő testekre nem hat centrifugális erő.
                \item Az $F_{ts}$ tapadási súrlódási erő és a felületeket összenyomó $F_t$ erő kötött az alábbi összefüggés áll fenn: \underline{\textbf{$F_{ts} \le F_t \cdot \mu_0$}} ahol $\mu_0$ a \underline{\textbf{tapadási súrlódási együttható}}.
                \item Egy elütött jégkorong lassulásának nagysága $0,5 m/s^2$. A jég és a korong közöti csúszási súrlódási együttható értéke közelítőleg: \underline{\textbf{$ 0,05 \;\;\;\; F_s=ma \;\;\;\; a=g\mu \;\;\;\; u=\frac{a}{g}$}}.
                \item A Föld déli féltekén  északi irányban közlekedő vonatokra \underline{\textbf{nyugati irányban}} mutató Coriolis-erő hat.
                \item Lefelé gyorsuló liftben a lifthez képest nyugvó test súlya \underline{\textbf{kissebb}}, mint a testre ható gravitációs erő.
                \item Konzervatív erőtér munkája nem függ az erőtérben mozgó test által megtett úttól, csak a mozgás \underline{\textbf{kezdő- és végpontjának}} helyzetétől.
                \item A Föld gravitációs erőterébe helyezett test potenciális energiája akkor a legnagyobb, ha a testet \underline{\textbf{egy végtelen távoli pontba}} helyezzük.
                \item Egy sportoló $h$ magasságban emel egy $m$ tömegű súlyzót, majd visszateszi oda, ahonnan elvette. A sportoló nehézségi erőtér ellenében végzett munkája \underline{\textbf{nulla}}.
                \item Egy körmozgás sugarát és szögsebességét is megduplázzuk. A körmozgást végző test centripetális gyorsulása \underline{\textbf{8}}-szorosára/-szeresére nő.
            \end{enumerate}}
        
    \subsection{2018.12.19. - Vizsga}

        \sloppy{
            \begin{enumerate}
                \item A testek mozgásállapot változtató hatás ellenében tanúsított ellenállást a \underline{\textbf{tömeg}} nevű fizikai mennyiséggel jellemezzük.
                \item Rugalmas ütközés esőtt a testek mechanikai energiáinak összege mindig \underline{\textbf{ugyanakkora}} mint ütközés után.
                \item Az olyan \underline{\textbf{vonatkoztatási rendszereket}}, ahol igaz a tehetetlenség törvénye, inerciarendszereknek nevezzük.
                \item Egyenletes körmozgás esetén a sebességvektor \underline{\textbf{nagysága}} nem változik.
                \item Tömegpontrendszer \underline{\textbf{impulzusa}} megmarad, ha a tömegpontrendszerre ható külsp erők eredője nulla.
                \item Az impulzusmomentum-tétel matematikai alakja a következő: \underline{\textbf{$\overline{M}=\overline{N} \;\;\;\; \overline{M} = \stackrel[\Delta t \to 0]{}{\sin}  \frac{\overline{\Delta N}}{\Delta t}$}}, ahol \textbf{\textit{M}} pontrendszerre ható külső erők forgatónyomatéka, \textbf{\textit{N}} pedig a pontrendszer impulzusmomentuma.
                \item Kepler I. törvénye értelmében a bolygók \textcolor{red}{\underline{\textbf{ellipszispályán keringenek, egyik fákunpontban a nap áll.}}}.
                \item Egy fizikai inga tömegközéppontja igen közel esik a felfüggesztési tengelyhez. Ebben az esetben az inga lengésideje igen \underline{\textbf{nagy}}.
                \item A munkatétel értelmében a testre ható erők munkája egyenlő a test \underline{\textbf{kinetikus energiájának megváltozásával}}.
                \item A pörgettyűk impulzusmomentum-vektorának külső erők hatására bekövetkező irányváltozását \textcolor{red}{\underline{\textbf{valmai}}} nevezzük.
                \item \textcolor{red}{\underline{\textbf{valami}}} hullámokban a közeg rezgéseinek kitérése párhuzamos a hullám terjedési irányával.
                \item \underline{\textbf{Izobár}} folyamatokban a gáz térfogata egyenesen atányos a hőmérséklettel.
                \item Az \textcolor{red}{\underline{\textbf{valami}}} tételének értelmében a gázrészecskék egyes szabadsági fokaira jutó átlagos energia egyenlő.
                \item A gáz által végzett munka egy körfolyamat során egyenlő a $P-V$ síkon ábrázolt folyamatgörbe \underline{\textbf{által határolt területtel}}.
                \item Az \underline{\textbf{extenzív}} állapotjellemzők kölcsönhatás során összeadódnak.
            \end{enumerate}}

    \underline{\textbf{}}

\end{document}