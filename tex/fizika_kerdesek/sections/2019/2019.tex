\documentclass[../../fizika_kerdesek.tex]{subfiles}

\begin{document}

    \subsection{2019.11.14. - Nagy ZH}

        \sloppy{
            \begin{enumerate}
                \item Az SI mértékrendszer kilogramm alapegységét korábban tömegetalonhoz rögzítették. 2019 májusa óta azonban az alapegységeket \underline{\textbf{}} -hoz rögzítik. 
                \item A pillanatnyi gyorsulás a  \underline{\textbf{}} függvény érintőjének meredekségével egyezik meg. 
                \item Ha egy függőleges hajítás kezdősebességét megduplázzuk, a pálya tetőpontjának magassága  \underline{\textbf{}} szeresére nő. 
                \item Ferdén elhajított test gyorsulásvektora, valamint sebességvektora által bezárt szög az idő függvényében monoton  \underline{\textbf{}}.
                \item Newton III. törvénye értelmében két kölcsönhatásba lépő tömegpont  \underline{\textbf{}} erővel hat egymásra. 
                \item Légüres térben azonos magasságból ejtett különböző anyagú testek egyszerre érne földet. A testek \underline{\textbf{}} és \underline{\textbf{}} tömegének aránya tehát anyagfüggetlen.
                \item Ha egy lejtő halásszöge tart 90 fok-hoz, a lejtőn lecsúszó test gyorsulása \underline{\textbf{}} tart.
                \item Szabadon eső test kinetikus energiája az esési idő \underline{\textbf{}} hatványával arányos. 
                \item Guruló autó a közegellenállás hatására idővel megáll. A közegellenállási erő munkája \underline{\textbf{}} előjelű.
                \item Potenciális energiát csak akkor definiálhatunk egy erőtérben, ha az \underline{\textbf{}}.
                \item Ismerjük egy adott rugó által kifejtett F(x) erőt az x megnyúlás függvényében. A rugóban tárolt energia meghatározható az F(X) függvény \underline{\textbf{}} kiszámításával. 
                \item A tömegpontra ható erők eredője megegyezik a tömegpotn \underline{\textbf{}} változási gyorsaságával.
            \end{enumerate}}

    \subsection{2019.01.09. - Vizsga}

        \sloppy{
            \begin{enumerate}
                \item Ha egy tömegpontra ható erők \underline{\textbf{eredője nulla}}, a tömegpont mozgásállapota nem változik meg. 
                \item A talaj felett h magasságban v0 kezdősebességgel elhajítunk egy testet. A test akkor ér a leghamarabb földet, ha a sebesség iránya \underline{\textbf{függőlegesen felfelé mutat}}.
                \item A szabadon eső test gyorsulása akkor tekinthető állandónak, ha a nehézségi erőteret \underline{\textbf{homgénnek}} tekintjük. 
                \item Függőleges síkú körmozgást végző tömegpont pályájának \underline{\textbf{legalsó}} pontján a centripetális gyorsulás és a nehézségi erő vektora párhuzamos, de ellentétes irányú. 
                \item Rugalmas ütközések során a \underline{\textbf{mechanikai energia}} megmarad, hiszen az ütközéskor fellépő rugalmas erő konzervatívnak tekinthető.
                \item Tömegpontrendszer teljes impulzusmomentuma megmarad, ha a tömegpontrendszerre ható külső \underline{\textbf{erők forgatónyomatéka nulla}}.
                \item A bolygók ellipszis pályán keringenek. Ha a pálya alakja kör, az ellipszis két fókuszpontja \underline{\textbf{egy pontba esik}}.
                \item A Föld Naphoz viszonyított sebessége télen nagyobb, mint nyáron. A Kepler II. törvénye értelmében tehát a Nap-Föld távolság  télen \underline{\textbf{kisebb}}, mint nyáron. 
                \item Egy gerjesztett rezgés rezonancia frekvencián mérhető amplitúdója annál nagyobb, minél kisebb a rendszer \underline{\textbf{csillapítása}}.
                \item Az egydimenziós \underline{\textbf{}} egyik megoldása az $y(t)=A \sin(kx-wt)$ alakban felírható függvény, ahol k a \underline{\textbf{hullámszámot}} jelöli.
                \item A hullámszám a következőképp fejezhető ki a hullámhosszal: \underline{\textbf{$k=\frac{2\pi}{\lambda}$}}.
                \item Ideális gáz izoterm állapotváltozása során szükségszerű, hogy a gáz és a környezete között \underline{\textbf{hőcsere}} valósuljon meg.
                \item Hőszivattyúként dolgozó gáz körfolyamatát ábrázoljuk $P-V$ diagramon. A zárt görbe körüljárási iránya az óramutató járásával \underline{\textbf{ellentétes irányú}}.
                \item Egy gáz izobár mólhője mindig \underline{\textbf{nagyobb}}, mint az izochor mólhője. 
                \item A víz forrásponja \underline{\textbf{nő}}, ha a vízfelszín feletti gáztér nyomását növeljük.
            \end{enumerate}}

    \subsection{2019.01.16. - Vizsga}

        \sloppy{
            \begin{enumerate}
                \item Az egységnyi idő alatt bekövetkező \underline{\textbf{sebességváltozást}} gyorsulásnak nevezzük. 
                \item Ferdén elhajlított test pályájának \underline{\textbf{tetőpontján}} a sebesség vektora merőleges a gyorsulásvektorra.
                \item A vízszintes talajról ferdén elhajlított test kezdősebességét megduplázzuk. A levegőben töltött idő \underline{\textbf{kétszeresére nő}}.
                \item Függőleges tengely körül forgó edényben a folyadék felszínek \underline{\textbf{forgásparabiloid}} alakú.
                \item Egy rugótt 1 $J$ munka árán tudjuk nyújtatlan állapotához képest 1 cm-el megnyújtani. Ha tovább akarjuk nyújtani 1 cm-ről 2 cm-re, további \underline{\textbf{3 J}} munkát kell végeznünk. 
                \item \underline{\textbf{Konzervatív}} erőtérben mozgó tömegpont mechanikai energiája megmarad. 
                \item \underline{\textbf{Centrális}} erőtérben mozgó tömegpont impulzusmomentuma megmarad. 
                \item \underline{\textbf{A bolygópályák nagytengelyeinek köbei}} úgy aránylanak egymáshoz, mint a keringési idők négyzetei.
                \item A gerjesztés, valamint a gerjesztett rezgés közötti fáziskülönbség közelítőleg zérus, ha a gerjesztés frekvenciája \underline{\textbf{lényegesen kisebb}} mint a rezonancia-frekvencia.
                \item Állóhullám két ugyanolyan frekvenciájú, \underline{\textbf{ellentétes irányban}} terjedő hullám interferenciájaként alakul ki.
                \item Egyik végén zárt, másik végén nyitott síp alapharmonikusának hullámhossza \underline{\textbf{négysezrese}} a síp hosszának. 
                \item Az ideális gázok kinetikus elmélete szerint a gázrészecskék átlagos \underline{\textbf{energiája}} arányos a gáz hőmérsékletével. 
                \item Az adiabatikus állapotváltozásokat leíró $PV^k=$ \textit{állandó} összefüggésben a $k$ kitevő a gáz \underline{\textbf{izobár és izochor mólhőjének}} hányadosaként áll elő. 
                \item Halmazállapot változás során az anyagok hőt vesznek fel, vagy adnak le, de \underline{\textbf{hőmérsékletük}} mégsem változik. 
                \item A jég olvadáspontja \underline{\textbf{csőkken}} ha felületére nyomás nehezedik. 
            \end{enumerate}}

    \subsection{2019.01.21. - Vizsga}

        \sloppy{
            \begin{enumerate}
                \item Egy tömegpont gyorsulása arányos a \underline{\textbf{rá ható erők eredőjével}}.
                \item Vízszintesen elhajlított test gyorsulásvektora és sebességvektora \underline{\textbf{a földetérés}} pillanatában zárja be a legkisebb szöget egymással.
                \item Ferdén elhajlított test pályájának alakja \underline{\textbf{parabola}}.
                \item Egy asztalon nyugvó testre ható tartóerő ellenereje a \underline{\textbf{súlyerő}}.
                \item Föld felszínéhez közel, körpályán keringő műhold centripetális gyorsulása megegyezik a \underline{\textbf{gravitációs gyorsulással}}.
                \item A nehézségi erőtér konzervatív, hiszen az erőtér által egy tömegponton végzett munka csak \underline{\textbf{a mozgás kezdő- és végpontjainak helyzetétől}} függ.
                \item Pontrendszer \underline{\textbf{tömegközéppontjának}} gyorsulása arányos a pontrendszerre ható külső erők eredőjével.
                \item A \underline{\textbf{szökési sebesség}} megadja, mekkora kezdősebességgel kell egy tömegpontot indítani egy adott bolygó felszínéről, hogy az képes legyen a bolygótól végtelen messzire távolodni. 
                \item Egy tömegpont harmonikus rezgőmozgást végez, ha a rá ható erő \underline{\textbf{\textcolor{red}{???}}} de azzal ellentétes irányú.
                \item Két kismértékben eltérő frekvenciájú hanghullám interferenciájának eredményét \underline{\textbf{lebegésnek}} hívjuk.
                \item Egy mindkét végén nyitott síp egyik végét befogjuk. A síp alaphangjának frekvenciája \underline{\textbf{$\frac{1}{2}$}}-szeresére változik. 
                \item Az idális gázok kinetikus elmélete szerint a gázrészecskék egymással és az edény falával \underline{\textbf{rugalmasan}} ütköznek. 
                \item A P-V diagram tetszőleges pontján áthaladó adiabata, valamint izoterma görbék közül az \underline{\textbf{adiabaták}} a meredekebbek. 
                \item A 0 celzius fokos jég sűrűsége \underline{\textbf{kissebb}}, mint a $0$ celzius fokos vízé. 
                \item Ha egy adot tömegű anyagdarab adott mértékben történő felmelegítéséhez sok hő kell, az azt jelenti, hogy anyag \underline{\textbf{fejhője}} nagy. 
            \end{enumerate}}

    \underline{\textbf{}}

\end{document}