\documentclass[../../fizika_kerdesek.tex]{subfiles}

\begin{document}

    \subsection{2021.11.12. - Nagy ZH}

        \sloppy{
            \begin{enumerate}
                \item A mechanika jelenségeit az alábbi három SI alapmennyiségből származtatjuk: \underline{\textbf{hosszúság, tömeg, idő}}.
                \item Ha egy tömegpont sebesség-idő függvényének változási sebességét határozzuk meg, a \underline{\textbf{gyorsulás-idő függvényt}} kapjuk. 
                \item Vízszintes talaj fölött $h$ magasságból úgy kívánunk elhajítani egy testet adott $v$ kezdősebességgel, hogy az a legtovább tartózkodjon a levegőben. A kezdősebesség vektor iránya \underline{\textbf{függőlegesen felfelé mutató}}.
                \item Egy szabadon eső test sebesség-idő grafikonja egy \underline{\textbf{lineáris}} függvény. Az elejtett testek $v(t)$ grafikonja a gyakorlatban mindig az ideális görbe \underline{\textbf{alatt}} helyezkedik el a közegellenállás miatt. 
                \item Az \underline{\textbf{inerciarendszerek}} egymáshoz képest nyugalomban vannak, vagy egyenes vonalú egyenletes mozgást végeznek. 
                \item Az erők egy csoportját úgy definiáljuk, hogy hatásukra a tömegpont mozgása kielégítsen bizonyos kényszerfeltételeket. Ezek az erők a \underline{\textbf{kényszererők}}.
                \item Rögzített tengelyű, súrlódásmentes csigán átvetett, elhanyagolható tömegű, nyújthatatlan kötél nem változtatja meg a \underline{\textbf{kötélerő}} nagyságát, csupán az \underline{\textbf{erő irányát}} módosítja. 
                \item A munkatétel értelmében a tömegpontra ható erő \underline{\textbf{munkája egyenlő a tömegpont kinetikus energiá}}\\\underline{\textbf{-jának megváltozásával}}.
                \item Egy test \underline{\textbf{mechanikai energiája}} a test kinetikus és \underline{\textbf{potenciális}} energiáinak összege. 
                \item Egy tömegpont impulzusának idő szerinti deriváltja egyenlő a \underline{\textbf{tömegpontra ható erők eredőjével}}.
                \item Tömegpontrendszerek impulzusa állandó, ha a pontrendszerre \underline{\textbf{ható külső erők eredője nulla}}.
                \item Pontszerű test gravitációs terében a potenciális energia \underline{\textbf{fordítottan}} arányos a centrumtól mért távolsággal.
            \end{enumerate}}

    \subsection{2021.01.06.}

        \sloppy{
            \begin{enumerate}
                \item Egy test 2m utat tesz meg 1$m/s$ sebességgel, további 2m utat pedig 2$m/s$ sebességgel. A test átlagsebessége \underline{\textbf{4/3 m/s}}.
                \item Egy tömegpont \underline{\textbf{potenciális energiája}} megadja, mennyi munkavégzés árán juttatható a tömegpont egy adott referenciapontból a konzervatív erőtér kiszemelt pontjába.
                \item Egy $R$ sugarú, $m$ tömegű gyűrű tehetetlenségi nyomatéka a tömegközépponton átmenő, gyűrű síkjára merőleges tengelyre nézve \underline{\textbf{$mR^2$}}, a gyűrű kerületi pontján átmenő tengelyre vonatkoztatva \underline{\textbf{$2mR^2$}}.
                \item Egy körmozgás sugara 1m, periódusideje 4s. Az 1 másodperc alatt bekövetkező elmozdulás nagysága \underline{\textbf{$\sqrt{2}$ méter}}. 
                \item Függőlegesen felfelé elhajított test gyorsulása a pálya tetőpontján \underline{\textbf{ugyanakkora}}, mint az elhajítást követő pillanatban.
                \item Adott hajlásszögű lejtőn magára hagyott, tisztán gördülő golyó gyorsulása \underline{\textbf{nagyobb}}, mint egy ugyanakkora tömegű hengeré.
                \item Ha egy fizikai ingát a tömegközéppontjához igen közel függesztünk fel, a lengésidő határértékben tart \underline{\textbf{végtelenhez}}.
                \item Rugóra függesztett rezgő test éppen átmegy az egyensúlyi helyzetén. Gyorsulásvektorának nagysága ebben az esetben \underline{\textbf{nulla}}.
                \item Egy test mozgását az "$ma+bv+kx = F(t)$" egyenlet írja le, ahol $F(t)$ egy szinuszosan változó külső erő. A test ekkor \underline{\textbf{gerjesztett harmonikus rezgést}} végez.
                \item A Föld felszínéről indított test szökési sebességét a(z) \underline{\textbf{mechanikai energia megmaradás}} tétele segítségével számolhatjuk ki.
                \item Az \underline{\textbf{ekvipartíció}} tétele értelmében egy részecskerendszer teljes energiájának meghatározásához szükséges változók mindegyikéhez $kB\frac{T}{2}$ átlagenergia tartozik.
                \item Ha a kifeszített húron szembe haladó két azonos frekvenciájú hullám állóhullámot hoz létre, akkor a zérus kitérésű helyeket \underline{\textbf{csomópontoknak}} nevezzük.
                \item Ha egy folyamat során a rendszer entrópiája növekszik, akkor biztos, hogy a folyamat \underline{\textbf{irrevirzibilis}}.
                \item A hűtőszekrény által felhasznált munka 200 J, a teljesítménytényezője 6. A hűtőszekrény belsejéből elvont hő ekkor: \underline{\textbf{1200 J}}.
                \item Az Adiabatikus folyamatok során a $P \cdot V^k$ szorzat állandó. A térfogat kitevőjében szereplő k konstans az \underline{\textbf{izobár és az izochor mólhő}} hányadosa.
            \end{enumerate}}

    \subsection{2021.01.12.}

        \sloppy{
            \begin{enumerate}
                \item A tiszta gördülés feltétele, hogy \underline{\textbf{a kerék talajjal érintkező pontja}} zérus sebességű legyen.
                \item Forgó vonatkoztatási rendszerben csak akkor lép fel Coriolis-erő, ha a test vonatkoztatási rendszerhez képesti sebességvektora \underline{\textbf{nem nulla, és nem párhuzamos a rendszer forgástengelyével}}. 
                \item Rugalmas ütközés során csak \underline{\textbf{konzervatív}} erők lépnek fel, ezért érvényes a mechanikai energia megmaradás törvénye.
                \item Két test egydimenziós tökéletesen rugalmatlan ütközése után a két test sebessége \underline{\textbf{megegyezik}}.
                \item Pontrendszer tömegközéppontjának mozgásállapotát csak \underline{\textbf{külső erők}} változtathatják meg.
                \item Adott bolygó felszínén a II. kozmikus sebesség \underline{\textbf{$\sqrt{2}$-ször}} akkora, mint az I. kozmikus sebesség.
                \item Egy kisbolygó pályájának nagytengelye 8-szor nagyobb, mint a Föld-Nap távolság. A kisbolygó keringési ideje \underline{\textbf{8}} év.
                \item Egy szivacsos szerkezetű, gömb alakú, $\rho$ átlagsűrűségű kisbolygó napközelben megolvad, és tömör, $2\rho$ sűrűségű gömbbé sűrűsödik össze anyagveszteség nélkül. A bolygó felszínén a gravitációs gyorsulás értéke \underline{\textbf{$\sqrt[3]{4}$ vagy $2^{\frac{2}{3}}$}} szorosára nő.
                \item Egy tömegpontrendszer \underline{\textbf{impulzusmomentuma}} akkor marad meg, ha a pontrendszerre ható külső erők forgatónyomatéka nulla.
                \item Ha egy pörgettyű tengelyét egy ponton rögzítjük úgy, hogy az nem esik egybe a tömegközépponttal, a tengely mozgása egy kúppalást felületét súrolja. A jelenséget \underline{\textbf{precessziónak}} nevezzük.
                \item Egy egyenlítői vulkánkitörés következtében az $R$ sugarú Föld középpontjából $m$ tömegű láva ömlik a felszínre. A Föld tehetetlenségi nyomatéka \underline{\textbf{$mR^2$}} értékkel növekedett meg.
                \item Az egydimenziós hullámegyenlet szerint a hullámfüggvény hely szerinti második deriváltja arányos a hullámfüggvény \underline{\textbf{idő szerinti második deriváltjával}}.
                \item Egy hőerőgépben lezajló körfolyamatot $P-V$ diagramon ábrázolva olyan görbét kapunk, melyeknek körüljárási iránya az óramutató járásával \underline{\textbf{megegyező irányú}}.
                \item Egy Carnot-gép hideg hőtartálya 0 ${}^\circ C$ fokos, a gép hatásfoka 50\%. A gép meleg hőtartálya \underline{\textbf{273}} ${}^\circ C$ fokos.
                \item Egy \underline{\textbf{egyatomos gáz}} részecske szabadsági fokainak száma három.
                \item A \underline{\textbf{kinetikus gázelmélet/ideális gázmodell}} felállításakor feltételezzük, hogy a gázrészecskék egymással és az edény falával tökéletesen rugalmasan ütköznek.
            \end{enumerate}}

    \subsection{2021.11.24. - Pót Nagy ZH}

        \sloppy{
            \begin{enumerate}
                \item Az SI rendszerben a hosszúság, a \underline{\textbf{$\dots\dots\dots$}} és az idő alapmennyiségek, míg a sebesség egy \underline{\textbf{$\dots\dots\dots$}} mennyiség. 
                \item A sebesség-idő függvény meghatározható a gyorsulás-idő függvény \underline{\textbf{$\dots\dots\dots$}} bmeghatározásával.
                \item Egy ferdén elhajított test pillanatnyi sebességvektora és gyorsulásvektora a \underline{\textbf{$\dots\dots\dots$}} zárja be a legkisebb szöget egymással.
                \item A talajról függőlegesen feldobott test 30 m/s sebességgel esett le. A test körülbelül \underline{\textbf{$\dots\dots\dots$}} másodpercig tartózkodott a levegőben.
                \item Newton II. törvénye értelmében egy tömegpont \underline{\textbf{$\dots\dots\dots$}}.
                \item Egy bolygó naptávolban 3-szor távolabb van a naptól, mint napközelben. A bolygó centripetális gyorsulásának maximuma és minimuma közti arány: \underline{\textbf{$\dots\dots\dots$}}.
                \item Egy $\alpha$ hajlásszögű lejtőn elhelyezett m tömegű test nem csúszik meg. A testre ható tapadási súrlódási erő értéke: \underline{\textbf{$\dots\dots\dots$}}
                \item A munkatétel értelmében egy tömegpont \underline{\textbf{$\dots\dots\dots$}} egyenlő a tömegpontra ható erők munkájával.
                \item A közegellenállási erő a sebesség négyzetével arányos. A közegellenállási erő teljesítménye a sebesség \underline{\textbf{$\dots\dots\dots$}} arányos.
                \item Egy forgó vonatkoztatási rendszerben akkor nem hat \underline{\textbf{$\dots\dots\dots$}} egy testre, ha az a forgó rendszer tengelyével párhuzamosan mozog.
                \item Egy R sugarú bolygó felszínén a potenciális energia értéke E. A bolygó felszíne felett 2R távolságra a potenciális energia értéke \underline{\textbf{$\dots\dots\dots$}}.
                \item Nehézségi erőtérben a potenciális energiát konvencionálisan az E=mgh összefüggéssel adjuk meg. Ilyenkor feltételezzük, hogy a nehézségi erőtér \underline{\textbf{$\dots\dots\dots$}}.
            \end{enumerate}}

    %\underline{\textbf{}}

\end{document}