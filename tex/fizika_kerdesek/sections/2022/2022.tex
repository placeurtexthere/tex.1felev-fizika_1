\documentclass[../../fizika_kerdesek.tex]{subfiles}

\begin{document}

    \subsection{2022.01.08.}

        \sloppy{
            \begin{enumerate}
                \item A Föld Naphoz viszonyított sebessége télen nagyobb, mint nyáron, tehát a Föld-Nap távolság télen \underline{\textbf{kisebb, mint nyáron}}.
                \item Gerjesztett rezgés amplitúdója rezonancia-frekvencián annál \underline{\textbf{nagyobb}} minél kisebb a rendszer csillapítása.
                \item Egy rezgés túlcsillapított, ha a sajátfrekvencia \underline{\textbf{kisebb}}, mint a csillapítás.
                \item A hullámszám fordítottan arányos a \underline{\textbf{hullámhosszal}}.
                \item Függőleges tengelyű, egyenletes körmozgást végző tömegpont gyorsulása és a nehézségi gyorsulás \underline{\textbf{90}} fokos szöget zár be egymással.
                \item Egy testet függőlegesen elhajítunk a talajról $\frac{v}{2}$ kezdősebességgel, egy másikat $45^\circ$ fokos szög alatt $v$ sebességgel. A \underline{\textbf{függőlegesen}} elhajított test ér földet hamarabb.
                \item A \underline{\textbf{Hooke}} törvény értelmében a rugó megnyúlása és a rugóerő között \underline{\textbf{lineáris}} kapcsolat van.
                \item Egy tömegpont mozgási energiájának megváltozása egyenlő a \underline{\textbf{tömegpontra ható erők mechanikai munkájával}}.
                \item \underline{\textbf{Centrális}} erőtérben mozgó tömegpont impulzusmomentuma megmarad.
                \item Egy mindkét végén nyitott síp alaphangját szólaltatjuk meg. Befogjuk a síp egyik végét. Az alaphang frekvenciája \underline{\textbf{1/2}} szeresére változik.
                \item Pontrendszer impulzusmomentumának idő szerinti deriváltja egyenlő \underline{\textbf{a pontrendszerre ható külső erők eredő forgatónyomatékával}}.
                \item A \underline{\textbf{centrifugális}} erő arányos a vonatkoztatási rendszer szögsebességének négyzetével.
                \item Az univerzális gázállandó és az Avogadro-szám hányadosa a \underline{\textbf{Boltzmann-állandó}}.
                \item Egy fekete test egységnyi felületén \underline{\textbf{kisugárzott hőteljesítmény}} arányos a test hőmérsékletének 4. hatványával.
                \item Egy hideg és egy meleg gáztartályt összenyitunk, a gázok összekeverednek. A rendszer \underline{\textbf{entrópiája}} növekedett.
                \item Egy gázrészecske átlagos \underline{\textbf{kinetikus energiája}} arányos a gáz hőmérsékletével.
            \end{enumerate}}

    \subsection{2022.01.08.}

        \sloppy{
            \begin{enumerate}
                \item A hely-idő függvény meredekség-függvénye a tömegpont \underline{\textbf{sebesség-függvényét}} adja meg.
                \item Vízszintes talajról elhajítunk egy testet először függőlegesen, majd ferdén, ugyanakkora nagyságú kezdősebességgel. A függőlegesen elhajított test sebessége földetéréskor \underline{\textbf{ugyanakkora}}, mint a ferdén elhajított testé.
                \item Ha egy testet kétszer magasabb toronyból ejtünk le, a földetéréskor mért sebessége \underline{\textbf{$\sqrt{2}$}}-szeresére nő.
                \item Egy tartálykocsi vízszintes talajon $g$ gyorsulással egyenletesen gyorsul. A folyadék felszíne a talajjal \underline{\textbf{$45^\circ$}} szöget zár be.
                \item Ismerjük egy rugó által kifejtett $F(x)$ erő nagyságát az $x$ megnyúlás függvényében. A rugó megnyújtásához szükséges munka kiszámítható az $F(x)$ függvény \underline{\textbf{görbe alatti területének}} kiszámításával.
                \item Az egyenlítőn észak felé haladó járműre ható Coriolis-erő \underline{\textbf{zérus}}.
                \item Centrális erőtérben mozgó tömegpont \underline{\textbf{impulzusmomentuma}} állandó.
                \item Pontrendszer impulzusa állandó, ha \underline{\textbf{a pontrendszerre ható külső erők eredője nulla}}.
                \item Egy matematikai inga tömegét megduplázzuk. Az inga lengésideje \underline{\textbf{nem változik}}.
                \item Ha a hullámtér rezgéseinek kitérése merőleges a hullám terjedési irányára, a hullám \underline{\textbf{transzverzális}}.
                \item Lebegés akkor jön létre, ha két \underline{\textbf{eltérő frekvenciájú}} hullám találkozik egymással.
                \item A \underline{\textbf{hőmérséklet, nyomás}} egy intenzív állapothatározó.
                \item Egy kétatomos gázmolekula szabadsági fokainak száma \underline{\textbf{5}}, ha a két atomot összetartó kémiai kötést merev rúdnak tekintjük.
                \item Ha a folyadék felett csökkentjük a gáztér nyomását, a folyadék forráspontja \underline{\textbf{csökken}}.
                \item Egy melegebb test $Q$ hőt ad le, amelyet egy hidegebb test vesz fel. A rendszer összes entrópia-változása \underline{\textbf{pozitív}}.
            \end{enumerate}}

    \subsection{2022.01.15.}

        \sloppy{
            \begin{enumerate}
                \item Két vektor \underline{\textbf{vektoriális szorzatának}} nagysága arányos a két vektor által kifeszített parallelogramma területével.
                \item Függőlegesen elhajított test esetén a földetérésig eltelt idő a kezdősebesség \underline{\textbf{első}} hatványával arányos.
                \item Videófelvételt készítünk egy szabadon eső testről. A felvételt feleakkora sebességgel, lassítva játszuk le. A filmen úgy tűnik, mintha a $g$ nehézségi gyorsulás az eredeti érték \underline{\textbf{1/4}}-szerese lenne. 
                \item Egy repülőgép függőleges síkú körpályán mozog, annak éppen a legalsó pontján tartózkodik. A centripetális gyorsulás, valamint a nehézségi erő vektora \underline{\textbf{ellentétes}} irányba mutat.
                \item A gravitációs tömegvonzás törvényében szereplő $\gamma$ gravitációs állandó SI mértékegysége: \underline{\textbf{$\frac{Nm^2}{kg^2}$}}.
                \item Pontrendszer tömegközéppontjának mozgásállapotát csak \underline{\textbf{külső}} erők változtatják meg.
                \item Tökéletesen rugalmas ütközéskor a mechanikai energia megmarad, mert az ütközéskor fellépő rugalmas erők \underline{\textbf{konzervatívak}}.
                \item Egy rugóból és tömegből álló rezgő rendszer rugóját középen kettévágjuk, és a fél rugóra akasztjuk vissza a tömeget. A rendszer sajátfrekvenciája \underline{\textbf{$\sqrt{2}$}}-szeresére változott.
                \item Egyik végén nyitott, másik végén zárt síp alaphangjának és első felharmonikusának frekvencia-aránya: \underline{\textbf{1:3}}.
                \item Egy gáz \underline{\textbf{nyomása}} az edény falának ötköző részecskék impulzusváltozásából származik.
                \item Egy ideális gáz \underline{\textbf{izobár és izochor}} mólhőjének különbsége az univerzális gázállandót adja.
                \item A P-V diagramon ábrázolt állapotváltozás \underline{\textbf{görbe alatti területe}} a gáz által végzett munkát adja meg.
                \item \underline{\textbf{Adiabetikus}} állapotváltozás során a gáz és környezete között nincs hőcsere.
                \item Egy Carnot-gép hatásfoka 50\%. A hideg hőtartály abszolút hőmérsékletét megfelezzük. A gép hatásfoka: \underline{\textbf{75\%}}.
            \end{enumerate}}

    \subsection{2022.01.22.}

        \sloppy{
            \begin{enumerate}
                \item Az inercia-rendszer olyan vonatkoztatási rendszer, melyben \underline{\textbf{igaz a tehetetlenség törvénye}}.
                \item Origóból $45^\circ$-os szög alatt elhajított test pályájának tetőpontján a helyvektor $y$ koordinátája \underline{\textbf{kisebb}}, mint az $x$ koordinátája.
                \item Egy ismeretlen bolygó feszínén a nehézségi gyorsulás értéke fele a földi értéknek. A bolygón adott magasságból elejtett test földetérési ideje \underline{\textbf{$\sqrt{2}$}}-szerese a Földön mért földetérési időnek.
                \item Egyenletesen gyorsuló körmozgást végző test eredő gyorsulásvektora és sebességvektora által bezárt szög \underline{\textbf{kiebb}} mint $90^\circ$.
                \item Az Északi-sarkon nyugvó testre nem hat \underline{\textbf{centrifugális}} erő.
                \item Egy homogén tömegeloszlású, gömb alakú bolygó felszínén a nehézségi gyorsulás értéke $g$. Egy kétszer akkora, ugyanilyen anyagú bolygó felszínén a nehézségi gyorsulás értéke: \underline{\textbf{$2g$}}.
                \item A Napból a bolygóhoz húzott sugár \underline{\textbf{egyenlő időközök alatt egyenlő területeket}} súrol.
                \item Pontrendszer \underline{\textbf{impulzusa}} állandó, ha a pontrendszerre ható külső erők eredője nulla.
                \item A csillapítási tényző SI mértékegysége: \underline{\textbf{$\frac{1}{s}$}}.
                \item A mindkét végén nyitott síp alaphangjának és első felharmonikusának frekvencia-aránya: \underline{\textbf{1:2}}.
                \item Az ekvipartíció tételének értelmében a részecskék egy szabadségi fokára jutó energia átlagos értéke: \underline{\textbf{$\frac{1}{2}kT$}}.
                \item Ha a Boltzmann-állandót és az Avogadro-számot összeszorozzuk, az \underline{\textbf{univerzális gázállandót}} kapjuk.
                \item A $P-V$ diagram adott pontján áthaladó izotermia-görbe meredekségének abszolút értéke \underline{\textbf{kisebb}}, mint az ugyanazon ponton áthaladó adiabata-görbéé.
                \item Egy test belső energiájának megváltozása egyenlő a testtel közölt hő, valamint a \underline{\textbf{testen végzett munka}} összegével.
                \item A hőszivattyúk $P-V$ diagramon ábrázolt körfolyamatának körüljárási iránya az óramutató járásával \underline{\textbf{ellentétes}} irányú.
            \end{enumerate}}

    \subsection{2022.01.29.}

        \sloppy{
            \begin{enumerate}
                \item A kinematika a \underline{\textbf{mozgások}} leírásával foglalkozó tudományág.
                \item Függőlegesen elhajítunk egy testet. A közegellenállás miatt bekövetkező mechanikai energiaveszteség az emelkedési szakaszban \underline{\textbf{nagyobb}}, mint a süllyedés során.
                \item A Föld egyenlítői átmérője nagyobb, mint a pólusokat összekötő átmérő. Ennek oka a Föld tömegpontjaira ható \underline{\textbf{centrifugális erő}}.
                \item Egy kanyarban fékező jármű gyorsulásvektora és sebességvektora által bezárt szög \underline{\textbf{nagyobb}} mint $90^\circ$.
                \item Az árapály jelenséget a \underline{\textbf{Hold gravitációs tömegvonzása}} okozza.
                \item Egy kisbolygó kétszer nagyobb sugarú körpályán kering a Nap körül, mint a Föld. A kisbolygó keringési \underline{\textbf{$\sqrt{8}$}}.
                \item Két egyforma méretű bolygó egyike kétszer akkora sűrűségű anyagból van, mint a másik. A sűrűbb bolygó felszínén a szökési sebesség \underline{\textbf{$\sqrt{2}$}}-szer akkora, mint a ritkább bolygón.
                \item Pontrendszer \underline{\textbf{impulzusmomentuma}} állandó, ha a pontrendszerre ható külső erők forgatónyomatékainak eredője nulla.
                \item Két eltérő frekvenciájú rezgés szuperpozíciójakor kialakuló lebegés frekvenciája a két rezgés frekvenciájának \underline{\textbf{különbségével}} arányos.
                \item Hullámvezető zárt végéről visszaverődő harmonikus hullám \underline{\textbf{$\pi$}} fázisugrást szenved.
                \item Alulcsillapított harmonikus rezgés amplitúdója \underline{\textbf{$e^{-pt}$}} függvény szerint csökken.
                \item A Boltzmann-állandó SI mértékegysége: \underline{\textbf{$\frac{J}{K}$}}.
                \item Gáz izobár tágulása során felvett hő a gáz belső energiájának növekedésére, és a \underline{\textbf{gáz munkavégzésére}} fordítódik.
                \item A $P-T$ diagramon azt a pontot nevezzük hármaspontnak, ahol az adott anyag \underline{\textbf{három halmazállapota}} egyszerre fordulhat elő.
                \item A Carnot körfolyamat \underline{\textbf{izoterm}} és \underline{\textbf{adiabatikus}} állapotváltozásokból tevődik össze.
            \end{enumerate}}

    \subsection{2022.11.10. - Nagy ZH}

        \sloppy{
            \begin{enumerate}
                \item Az erő mértékegysége SI alapegységek segítségével kifejezve: \underline{\textbf{$\frac{kg\cdot m}{s^2}$}}.
                \item A mozgás kezdő- és végpontja közti pályagörbe hosszát \underline{\textbf{útnak}} nevezzük.
                \item A ferdén elhajított test függőleges tengelyre vetített mozgása \underline{\textbf{függőleges hajításnak felel meg}}.
                \item Ha egy testet kétszer magasabbról ejtünk le, az esési idő \underline{\textbf{$\sqrt{2}$}}-szeresére nő.
                \item Vízszintes úton gépkocsi gyorsít. Az autót a \underline{\textbf{tapadási súrlódási}} erő gyorsítja.
                \item Ugyanazon fékberendezés a kétszer nagyobb sebességgel mozgó járművet \underline{\textbf{4}}-szer hosszabb úton fékezi le, és állítja meg.
                \item Egy biciklis $v$ sebességgel mozog nyugvó közegben, miközben $P$ teljesítménnyel dolgozik a közegellenállás leküzdésére. Hirtelen $v$ sebességű szembeszél támad. A talajhoz viszonyított $v$ sebességének fenntartásához \underline{\textbf{4P}} teljesítmény szükséges.
                \item Lövedékkel deszkába lövünk. A deszka lövedéken végzett munkája \underline{\textbf{negatív}} előjelű.
                \item A nehézségi erőtérbe helyezett test potenciális energiájának megadására használt $E_{pot}=mgh$ összefüggés csak azon feltevés mellett érvényes, ha a nehézségi erőteret \underline{\textbf{homogénnek}} tekintjük.
                \item Egy tisztán gördülő roller első kereke kétszer akkora, mint a hátsó. Az első kerék kerületi pontjainak centripetális gyorsulása \underline{\textbf{$\frac{1}{2}$}}-szerese, mint a hátsó keréké.
                \item A tömegpontra ható \underline{\textbf{erők eredőjének munkája}} egyenlő a tömegpont kinetikus energiájának megváltozásával.
                \item Ha a pontrendszerre ható külső erők eredője zérus, a \underline{\textbf{pontrendszer impulzusa állandó}}.
            \end{enumerate}}

    \subsection{2022.11.23. - Pót Nagy ZH}

        \sloppy{
            \begin{enumerate}
                \item Az SI alapmennyiségek egységei garantáltan mindig újra reprodukálhatóak, mert értékeik \underline{\textbf{természeti állandókhoz}} vannak rögzítve.
                \item Az \underline{\textbf{egyenletes körmozgás}} mozgás sebességének és gyorsulásának nagysága is állandó, irányuk viszont folyamatosan változik.
                \item Egy vízszintes talajról induló ferde hajítás esetén a test gyorsulásvektora és sebességvektora 60 fokos szöget zár be egymással közvetlenül a becsapódás előtt. A hajítás kezdősebesség-vektora \underline{\textbf{$30^\circ$}} szöget zárt be a vízszintessel.
                \item Egy súrdásmentes lejtőn lecsúszó test gyorsulása $5 \frac{m}{s^2}$. A lejtő hajlásszöge körülbelül \underline{\textbf{$30^\circ$}} fokos.
                \item Egy $m$ tömegű test vízszintes felületen nyugszik. A testet $F$ erővel húzzuk vízszintesen, ám az nem mozdul. A felület és a test között a tapadási súrlódási együttható értéke $\mu_0$. A testre ható tapadási súrlódási erő nagysága \underline{\textbf{F}}.
                \item Egy rugó csak akkor fejti ki a megnyúlásával arányos nagyságú erőt, ha feltételezzük, hogy érvényes rá \underline{\textbf{Hooke}} törvénye.
                \item kg, m, és s alapegységekkel kifejezve 1 watt = 1 \underline{\textbf{$\frac{kgm^2}{s^3}$}}.
                \item Az elektromos fogyasztásmérő által használt $1 kWh$ energiaegysége \underline{\textbf{3600000}} Joule energiával egyezik meg. 
                \item Egy szabadon eső test két másodpercig zuhan. A nehézségi erőtér \underline{\textbf{3X}} annyi munkát végzett a testen a második másodpercben, mint az első másodpercben. 
                \item \underline{\textbf{Konzervatív}} erőtérben mozgó tömegpont mechanikai energiája megmarad.
                \item A Földön ásványkincseket bányászunk, és azokat elszállítjuk a Holdra. A két égitest közötti gravitációs vonzás ennek hatására \underline{\textbf{nő}}.
                \item Az egyenlítő felett átrepül egy repülőgép északról délre. A repülőre ható Coriolis-erő \underline{\textbf{nulla}}.
            \end{enumerate}}


\end{document}