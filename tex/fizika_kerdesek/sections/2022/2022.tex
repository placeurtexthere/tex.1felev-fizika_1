\documentclass[../../fizika_kerdesek.tex]{subfiles}

\begin{document}

    \subsection{2022.01.08.}

        \sloppy{
            \begin{enumerate}
                \item A Föld Naphoz viszonyított sebessége télen nagyobb, mint nyáron, tehát a Föld-Nap távolság télen \underline{\textbf{kisebb, mint nyáron}}.
                \item Gerjesztett rezgés amplitúdója rezonancia-frekvencián annál \underline{\textbf{nagyobb}} minél kisebb a rendszer csillapítása.
                \item Egy rezgés túlcsillapított, ha a sajátfrekvencia \underline{\textbf{kisebb}}, mint a csillapítás.
                \item A hullámszám fordítottan arányos a \underline{\textbf{hullámhosszal}}.
                \item Függőleges tengelyű, egyenletes körmozgást végző tömegpont gyorsulása és a nehézségi gyorsulás \underline{\textbf{90}} fokos szöget zár be egymással.
                \item Egy testet függőlegesen elhajítunk a talajról v/2 kezdősebességgel, egy másikat 45 fokos szög alatt v sebességgel. A \underline{\textbf{függőlegesen}} elhajított test ér földet hamarabb.
                \item A \underline{\textbf{Hooke}} törvény értelmében a rugó megnyúlása és a rugóerő között \underline{\textbf{lineáris}} kapcsolat van.
                \item Egy tömegpont mozgási energiájának megváltozása egyenlő a \underline{\textbf{tömegpontra ható erők mechanikai munkájával}}.
                \item \underline{\textbf{Centrális}} erőtérben mozgó tömegpont impulzusmomentuma megmarad.
                \item Egy mindkét végén nyitott síp alaphangját szólaltatjuk meg. Befogjuk a síp egyik végét. Az alaphang frekvenciája \underline{\textbf{1/2}} szeresére változik.
                \item Pontrendszer impulzusmomentumának idő szerinti deriváltja egyenlő \underline{\textbf{a pontrendszerre ható külső erők eredő forgatónyomatékával}}.
                \item A \underline{\textbf{centrifugális}} erő arányos a vonatkoztatási rendszer szögsebességének négyzetével.
                \item Az univerzális gázállandó és az Avogadro-szám hányadosa a \underline{\textbf{Boltzmann-állandó}}
                \item Egy fekete test egységnyi felületén \underline{\textbf{kisugárzott hőteljesítmény}} arányos a test hőmérsékletének 4. hatványával.
                \item Egy hideg és egy meleg gáztartályt összenyitunk, a gázok összekeverednek. A rendszer \underline{\textbf{entrópiája}} növekedett.
                \item Egy gázrészecske átlagos \underline{\textbf{kinetikus energiája}} arányos a gáz hőmérsékletével.
            \end{enumerate}}

    \subsection{2022.01.08.}

        \sloppy{
            \begin{enumerate}
                \item A hely-idő függvény meredekség-függvénye a tömegpont \underline{\textbf{sebesség-függvényét}} adja meg.
                \item Vízszintes talajról elhajítunk egy testet először függőlegesen, majd ferdén, egyanakkora nagyságú kezdősebességgel. A függőlegesen elhajított test sebessége földetéréskor \underline{\textbf{ugyanakkora}}, mint a ferdén elhajított testé.
                \item Ha egy testet kétszer magasabb toronyból ejtünk le, a földetéréskor mért sebessége \underline{\textbf{$\sqrt{2}$}}-szeresére nő.
                \item Egy tartálykocsi vízszintes talajon $g$ gyorsulással egyenletesen gyorsul. A folyadék felszíne a talajjal \underline{\textbf{$45^\circ$}} szöget zár be.
                \item Ismerjük egy rugó által kifejtett $F(x)$ erő nagyságát az $x$ megnyúlás függvényében. A rugó megnyújtásához szükséges munka kiszámítható az $F(x)$ függvény \underline{\textbf{görbe alatti területének}} kiszámításával.
                \item Az egyenlítőn észak felé haladó járműre ható Coriolis-erő \underline{\textbf{zérus}}.
                \item Centrális erőtérben mozgó tömegpont \underline{\textbf{impulzusmomentuma}} állandó.
                \item Pontrendszer impulzusa állandó, ha \underline{\textbf{a pontrendszerre ható külső erők eredője nulla}}.
                \item Egy matematikai inga tömegét megduplázzuk. Az inga lengésideje \underline{\textbf{nem változik}}.
                \item Ha a hullámtér rezgéseinek kitérése merpleges a hulllám terjedési irányára, a hullám \underline{\textbf{transzverzális}}.
                \item Legegés akkor jön létre, ha két \underline{\textbf{eltérő frekvenciájú}} hullám találkozik egymással.
                \item A \underline{\textbf{hőmérséklet, nyomás}} egy intenzív állapothatározó.
                \item Egy kétatomos gázmolekula szabadsági fokainak száma \underline{\textbf{5}}, ha a két atomot összetartó kémiai kötést merev rúdnak tekintjük.
                \item Ha a folyadék felett csökkentjük a gáztér nyomását, a folyadék forráspontja \underline{\textbf{csökken}}.
                \item Egy melegebb test $Q$ hőt ad le, amelyet egy hidegebb test vesz fel. A rendszer összes entrópia-változása \underline{\textbf{pozitív}}.
            \end{enumerate}}

    \subsection{2018.01.15.}

        \sloppy{
            \begin{enumerate}
                \item Két vektor \underline{\textbf{vektoriális szorzatának}} nagysága arányos a két vektor által kifeszített paralleloramma területével.
                \item Függőlegesen elhajított test esetén a földetérésig eltelt idő a kezdősebesség \underline{\textbf{első}} hatványával arányos.
                \item Videófelvételt készítünk egy szabadon eső testről. A felvételt feleakkora sebességgel, lassítva játszuk le. A filmen úgy tűnik, mintha a $g$ nehézségi gyorsulás az eredeti érték \underline{\textbf{1/4}}-szerese lenne. 
                \item Egy repülőgép függőleges síkú körpályán mozog, annak éppen a legalsó pontján tartózkodik. A centripetális csorsulás, valamint a nehézségi erő vektora \underline{\textbf{ellentétes}} irányba mutat.
                \item A gravitációs tömegvonzás törvényében szereplő $\gamma$ gravitációs állandó SI mértékegysége: \underline{\textbf{$\frac{Nm^2}{kg^2}$}}.
                \item Pontrendszer tömegközéppontjának mozgásállapotát csak \underline{\textbf{külső}} erők változtatják meg.
                \item Tökéletesen rugalmas ütközéskor a mechanikai energia megmarad, mert az ütközéskor fellépő rugalmas erők \underline{\textbf{konzervatívak}}.
                \item Egy rugóból és tömegből álló rezgő rendszer rugóját középen kettévágjuk, és a fél rugóra akasztjuk vissza a tömeget. A rendszer sajátfrekvenciája \underline{\textbf{$\sqrt{2}$}}-szeresére változott.
                \item Egyik végén nyitott, másik végén zárt síp alaphangjának és első felharmonikusának frekvencia-aránya: \underline{\textbf{1:3}}.
                \item Egy gáz \underline{\textbf{nyomása}} az edény felának ötköző részecskék impulzusváltozásából származik.
            \end{enumerate}}

    \subsection{}

        \sloppy{
            \begin{enumerate}
                \item 
            \end{enumerate}}

    \subsection{}

        \sloppy{
            \begin{enumerate}
                \item 
            \end{enumerate}}

    \subsection{}

        \sloppy{
            \begin{enumerate}
                \item 
            \end{enumerate}}

    \subsection{}

        \sloppy{
            \begin{enumerate}
                \item 
            \end{enumerate}}

    \subsection{}

        \sloppy{
            \begin{enumerate}
                \item 
            \end{enumerate}}

    \subsection{}

        \sloppy{
            \begin{enumerate}
                \item 
            \end{enumerate}}

    \subsection{}

        \sloppy{
            \begin{enumerate}
                \item 
            \end{enumerate}}

    \subsection{}

        \sloppy{
            \begin{enumerate}
                \item 
            \end{enumerate}}

    \subsection{}

        \sloppy{
            \begin{enumerate}
                \item 
            \end{enumerate}}

    \underline{\textbf{}}

\end{document}