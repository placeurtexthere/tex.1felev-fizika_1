\documentclass[../../fizika_kerdesek.tex]{subfiles}

\begin{document}

    \subsection{2023.01.11. - Vizsga}

        \sloppy{
            \begin{enumerate}
                \item \underline{\textbf{Az inercia}} rendszer olyan \underline{\textbf{vonatkoztatási}} rendszer, melyben érvényes a tehetetlenség törvénye. 
                \item Ferdén elhajlított tömegpont potenciális energiája akkor a legnagyobb, amikor a pillanatnyi sebességvektora \underline{\textbf{vízszintes}} irányú.
                \item Egy Holdon játszódó sci-fit forgatnak. A földi stúdióban felvett filmet, (melyen egy függőleges hajítás látható) \underline{\textbf{$\sqrt{6}$}}-szor lassabban kell levetíteni, hogy azt az illúziót keltse, mintha az a Holdon játszódna. ($g_{\text{Föld}}=6g_{\text{Hold}}$)
                \item Egyenletesen lassuló körmozgást végző test eredő gyorsulásvektora és sebességvektora által bezárt szög \underline{\textbf{nagyobb}} mint 90 fok. 
                \item A Déli-sarkon nyugvó testre nem hat \underline{\textbf{centrifugális}} erő.
                \item Egy bolygó tömege 16-szor akkora, mint a Földé, sugara pedig 2-szer akkor, mint a Földé. A bolygó felszínén a gravitációs gyorsulás értéke \underline{\textbf{4}}-szer akkor, mint a Földön.
                \item A Napból a bolygóhoz húzott sugár \underline{\textbf{azonos időközök alatt azonos területeket}} súrol. 
                \item Pontrendszer \underline{\textbf{impulzusa}} állandó, ha a pontrendszerre ható külső erők eredője nulla.
                \item A csillapítási tényező SI mértékegysége: \underline{\textbf{$\frac{1}{s}$}}.
                \item A mindkét végén nyitott síp alaphangjának és első felharmonikusának frekvencia-aránya: \underline{\textbf{1:2}}.
                \item Egy pörgettyű felfüggesztési pontja, és tömegközéppontja nem esik egybe. A pörgettyű tengelye nem függőleges. A pörgettyű tengelye egy kúppalást mentén mozog. A mozgás neve \underline{\textbf{precesszió}}.
                \item Egy pontrendszer perdületének idő szerinti deriváltja (változása) egyenlő a \underline{\textbf{pontrendszerre ható külső erők eredő forgatónyomatékával}}.
                \item A $P-V$ diagram adott pontján áthaladó izoterma-görbe meredekségének abszolút értéke \underline{\textbf{kissebb}}, mint az ugyanazon ponton áthaladó adiabata-görbéé.
                \item Egy test belső energiájának megváltozása egyenlő a testtel közözlt hő, valamint a \underline{\textbf{testen végzett munka}} összegével.
                \item A hőszivattyúk $P-V$ diagramon ábrázolt körfolyamatának körüljárási iránya az óramutató járásával \underline{\textbf{ellentétes}} irányú.
            \end{enumerate}}

    \subsection{2023.01.18. - Vizsga}

        \sloppy{
            \begin{enumerate}
                \item Az erő mértékegysége az SI alapmennyiségek egységével kifejezve \underline{\textbf{$\sum gm/s^2$}}.
                \item Két test kölcsönhatása során a testek \underline{\textbf{azonos}} nagyságú, \underline{\textbf{ellentétes}} irányú erővel hatnak egymásra. 
                \item 20$m/s$ kezdősebességgel függőlegesen lefelé elhajlított test sebessége körübelül \underline{\textbf{2s}} múlva megduplázódik.
                \item A tapadási súrlódási erő \underline{\textbf{maximális értéke}} arányos a felületeket összenyomó erővel.
                \item Egy lejtőn csúszó testre ható nehézségi erő kétszer akkora, mint a rá ható tartóerő. A lejtő hajlásszöge \underline{\textbf{$60^\circ$}}.
                \item Leejtünk két testet. Az egyiken a nehézségi erő kétszer annyi idő alatt végez ugyanannyi munkát, mint a másikon egységnyi idő alatt. A két test tömegének aránya \underline{\textbf{1:4}}.
                \item Pontszerű test gravitációs terében helyezett tömegpont potenciális energiája arányos a vonzócentrumtól mért távolság \underline{\textbf{reciprokával}}.
                \item Egy forgó kerék szögsebességét megduplázzuk. Impulzusmomentuma \underline{\textbf{2}} szeresére nő.
                \item Egy ellipszispályán keringő bolygó mozgása során háromszor távolabb került a naptól. A bolygó nap középpontjára vonatkoztatott impulzusmomentuma \underline{\textbf{nem}} változott. 
                \item Egy krumplit kötőtűvel átszúrunk, majd a tűt vízszintes helyzetben rögzítjük úgy, hogy az tengelye körül könnyen elfordulhasson. A tengely és a krumpli tömegközéppontja közti távolság $x$. Minél kisebb $x$ értéke, a krumpli-inga lengésideje annál \underline{\textbf{nagyobb}}.
                \item Kényszerrezgés amplitúdója rezonancia esetén adott gerjesztés mellett annál nagyobb, minél kisebb a rezgő rendszer \underline{\textbf{csillapítása}}.
                \item A Föld forgásának kimutatására alkalmas nagy lengésidejű, kis csillapítású inga neve \underline{\textbf{eötvös inga}}.
                \item Ha egy ideális gázzal végrehajtott állapotváltozás során a gáz nyomása arányos a hőmérséklettel, a folyamat \underline{\textbf{izochor}}.
                \item A kinetikus gázelmélet szerint a gáz \underline{\textbf{nyomása}} az edény falával ütköző gázrészecskék impulzusvátlozásából származik.
                \item Egy gázt eredeti térfogatának felére összenyomtuk, a nyomása négyszeresére nőtt. A gáz hőmérséklete \underline{\textbf{2}} -szorosa/-szerese eredeti hőmérsékletének.
            \end{enumerate}}

    \subsection{2023.11.09. - Nagy ZH}

        \sloppy{
            \begin{enumerate}
                \item Az inercia-rendszer olyan vonatkoztatási rendszer, melyben \underline{\textbf{érvényes a tehetetlenség törvénye}}.
                \item A mértékegységeket kiegészítő mega-, kilo-, milli-, mikro- stb előtagokat \underline{\textbf{prefixumok}}-nak nevezzük. 
                \item Egy test 2m utat tesz meg $1m/s$ sebességgel, további 2m utat pedig $2m/s$ sebességgel. A test átlagsebessége \underline{\textbf{$\frac{4}{3} \frac{m}{s}$}}.
                \item Origóból 45 fokos szög alatt elhajlított test pályájának tetőpontján a helyvektor $y$ koordinátája \underline{\textbf{kisebb}} mint az $x$ koordinátája.
                \item Egy ismeretlen bolygó felszínén a nehézségi gyorsulás értéke fele a földi értéknek. A bolygón adott magasságból elejtett test földetérési ideje \underline{\textbf{$\sqrt{2}$}}-szerese a Földön mért földetérési időnek. 
                \item Egyenletesen lassuló körmozgást végző tömegpont eredő gyorsulásvektora és sebességvektora álta bezárt szög \underline{\textbf{nagyobb}} mint 90 fok. 
                \item Egy tömegpont \underline{\textbf{potenciális energiája}} megadja, mennyi munkavégzés árán juttatható a tömegpont egy adott referenciapontból a konzervatív erőtér kiszemelt pontjába. 
                \item Az Északi-sarkon nyugvó testre nem hat \underline{\textbf{centrifugális (Coriolis)}} erő. 
                \item Egy körmozgás sugara 1 m, periódusideje 4s. Az 1 másodperc alatt bekövetkező elmozdulás nagysága \underline{\textbf{$\sqrt{2}m$}}.
                \item Egy homogén tömegeloszlású, gömb alakú bolygó felszínén a nehézségi gyorsulás értéke g. Egy kétszer akkora, ugyanilyen anyagú bolygó felszínén a nehézségi gyorsulás értéke. \underline{\textbf{$2g$}}.
                \item Függőlegesen felfelé elhajlított test gyorsulása a pálya tetőpontján \underline{\textbf{ugyanakkora}} mint az elhajlítást követő pillanatban. 
                \item Egy $M$ tömegű pontszerű test gravitációs terében mozgatott $m$ tömegpont potenciális energiáját a vonzócentrumtól $r$ távolságra a \underline{\textbf{$-\gamma Mm/v$}} összefüggés adja meg. A nulla potenciálú pont a \underline{\textbf{végtelenben}} van.                 
            \end{enumerate}}

    \subsection{2023.11.23. - Pót Nagy ZH}

        \sloppy{
            \begin{enumerate}
                \item A mechanika jelenségeit leíró mennyiségeket az alábbi három SI alapmennyiségből származtatjuk: \underline{\textbf{hosszúság, tömeg, idő}}.
                \item Ha egy tömegpont sebesség-idő függvényének idő szerinti deriváltját állítjuk elő, a \underline{\textbf{gyorsulás-idő függvény}}-t kapjuk.
                \item Vízszintes talaj fölött $h$ magasságból úgy kívánunk elhajítani egy testet adott $v$ kezdősebességgel, hogy az a legtovább tartózkodjon a levegőben. A kezdősebesség vektor iránya \underline{\textbf{függőlegesen felfelé mutató}}.
                \item Egy szabadon eső test sebesség-idő grafikonja egy \underline{\textbf{lineáris}} függvény. Az elejtett testek $v(t)$ grafikonja a gyakorlatban mindig az ideális görbe \underline{\textbf{alatt}} helyezkedik el a közegellenállás miatt. 
                \item A közegellenállási erő a sebesség négyzetével arányos. A közegellenállási erő teljesítménye a sebesség \underline{\textbf{köbével}} arányos. 
                \item Az erők egy csoportját úgy definiáljuk, hogy hatásukra a tömegpont mozgása kielégítsen bizonyos kényszerfeltételeket. Ezek az erők a \underline{\textbf{kényszererők}}.
                \item Rögzített tengelyű, súrlódásmentes csigán átvetett, elhanyagolható tömegű, nyújthatatlan kötél nem változtatja meg a(z) \underline{\textbf{erő}} nagyságát, csupán az \underline{\textbf{irányát}} módosítja. 
                \item A munkatétel értelmében a tömegpontra ható erő \underline{\textbf{munkája egyenlő a tömegpont kinetikus energiá}} \underline{\textbf{-jának megváltozásával}}.
                \item Egy test \underline{\textbf{mechanikai energiája}} a test kinetikus és \underline{\textbf{potenciális}} energiáinak összege. 
                \item Egy $R$ sugarú bolygó felszínén a potenciális energia értéke $E$. A bolygó felszíne felett $2R$ távolságra a potenciális energia értéke \underline{\textbf{$E/3$}}.
                \item Tömegpontrendszerek impulzusa állandó, ha a pontrendszerre \underline{\textbf{ható külső erők eredője nulla}}.
                \item Nehézségi erőtérben a potenciális energiát konvencionálisan az $E=mgh$ összefüggéssel adjuk meg. Ilyenkor feltételezzük, hogy a nehézségi erőtér \underline{\textbf{homogén}}.
            \end{enumerate}}
            
    \underline{\textbf{}}

\end{document}